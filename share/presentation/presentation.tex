% Created 2015-05-11 Mon 09:18
\documentclass[presentation]{beamer}
\usepackage[utf8]{inputenc}
\usepackage[T1]{fontenc}
\usepackage{fixltx2e}
\usepackage{graphicx}
\usepackage{longtable}
\usepackage{float}
\usepackage{wrapfig}
\usepackage{rotating}
\usepackage[normalem]{ulem}
\usepackage{amsmath}
\usepackage{textcomp}
\usepackage{marvosym}
\usepackage{wasysym}
\usepackage{amssymb}
\usepackage{hyperref}
\tolerance=1000
\usetheme{Antibes}
\usecolortheme{seagull}
\author{Sam Gwydir \and Chris Findeisen \and Martin Fracker \\ Rafael Moreno \and Kyle Wilson}
\date{\textit{<2015-05-10 Sun>}}
\title{CSCE315 Final Project Presentation}
\hypersetup{
  pdfkeywords={},
  pdfsubject={},
  pdfcreator={Emacs 24.5.1 (Org mode 8.2.10)}}
\begin{document}

\maketitle
\begin{frame}{Outline}
\tableofcontents
\end{frame}


\section{DickGrayson}
\label{sec-1}
\begin{abstract}
In the final project students were tasked with creating a collection of tools
that allow a user to encrypt and decrypt messages using the RSA encryption
algorithm and embed and extract messages (either plaintext or ciphertext) from
BMP images and WAV audio files.
\end{abstract}

\begin{frame}[fragile,label=sec-1-1]{DickGrayson}
 \begin{block}{Introduction}
\emph{DickGrayson} is a collection of tools that allow a user to encrypt and decrypt
messages using the RSA encryption algorithm and embed and extract messages
(either plaintext or ciphertext) from BMP images and WAV audio files.
\end{block}

\begin{block}{Tools}
\begin{description}
\item[{\texttt{munchkincrypt}}] (aka \texttt{rsa-crypt}) RSA Encryption
\item[{\texttt{dorothy}}] (aka \texttt{rsa-attack}) RSA Attacks
\item[{\texttt{munchkinsteg}}] (aka \texttt{stego-crypt}) Steganography
\item[{\texttt{toto}}] (aka \texttt{stego-attack}) Steganography Attacks
\end{description}
\end{block}
\end{frame}

\section{Design}
\label{sec-2}
\begin{frame}[fragile,label=sec-2-1]{Design Decisions}
 \begin{block}{DickGrayson}
\begin{description}
\item[{Target OS}] Linux \texttt{x86\_64} (\texttt{build.tamu.edu})
\item[{Compiler}] \href{https://gcc.gnu.org/}{GCC 4.9.2}
\item[{Language}] C++14
\item[{Build System}] \href{https://cmake.org}{CMake}
\item[{Numerics}] \href{https://gmplib.org}{GNU Multiple Precision Library}
\item[{Unit Testing}] \href{https://code.google.com/p/googletest/}{Google Test}
\item[{Continuous Integration}] \href{http://travis-ci.org}{travis-ci}
\item[{Code Coverage}] \href{http://coveralls.io}{coveralls}
\end{description}
\end{block}
\end{frame}
\begin{frame}[label=sec-2-2]{Implementation Decisions}
\begin{block}{RSA Encryption}
\begin{itemize}
\item Especially, explain what attacks you chose to make for both RSA and stego, and
how your two stego schemes work
\item Show proofs of Test-Driven Development (screenshots of failing / passing
tests, GitHub revision history, running test code in the demo, etc)
\end{itemize}
\end{block}
\begin{block}{RSA Attacks}
\end{block}
\begin{block}{Steganography}
\end{block}
\begin{block}{Steganography Attacks}
\end{block}
\end{frame}

\section{Demonstration}
\label{sec-3}
\begin{frame}[label=sec-3-1]{RSA:}
\begin{block}{generate primes}
\end{block}
\begin{block}{generate public and private keys}
\end{block}
\begin{block}{encrypt and decrypt example messages}
\end{block}
\begin{block}{verify with openssl}
\end{block}
\end{frame}
\begin{frame}[label=sec-3-2]{RSA attacks}
\end{frame}
\begin{frame}[label=sec-3-3]{LSB:}
\begin{block}{Embedding message}
\end{block}
\begin{block}{Extracting message}
\end{block}
\end{frame}
\begin{frame}[label=sec-3-4]{Attack on LSB (at least 2)}
\end{frame}
\begin{frame}[label=sec-3-5]{ANS:}
\begin{block}{Very brief recap of how it works}
\end{block}
\begin{block}{Embedding message}
\end{block}
\begin{block}{Extracting message}
\end{block}
\end{frame}
\begin{frame}[label=sec-3-6]{Attack on ANS:}
\begin{block}{Detection of stego'ed objects}
\end{block}
\begin{block}{Disruption / destruction of hidden message}
\end{block}
\end{frame}
\section{Conclusions}
\label{sec-4}
\section{Future research directions (how your program could be improved or extended)}
\label{sec-5}
\section{References}
\label{sec-6}
\begin{frame}[label=sec-6-1]{Links}
\begin{description}
\item[{GCC 4.9.2}] \url{https://gcc.gnu.org}
\item[{GNU Multiple Precision Library}] \url{https://gmplib.org}
\item[{Google Test}] \url{https://code.google.com/p/googletest/}
\item[{travis-ci}] \url{https://travis-ci.org}
\item[{coveralls}] \url{https://coveralls.io}
\end{description}
\end{frame}
% Emacs 24.5.1 (Org mode 8.2.10)
\end{document}