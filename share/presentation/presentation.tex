% Created 2015-05-11 Mon 10:58
\documentclass[presentation]{beamer}
\usepackage[utf8]{inputenc}
\usepackage[T1]{fontenc}
\usepackage{fixltx2e}
\usepackage{graphicx}
\usepackage{longtable}
\usepackage{float}
\usepackage{wrapfig}
\usepackage{rotating}
\usepackage[normalem]{ulem}
\usepackage{amsmath}
\usepackage{textcomp}
\usepackage{marvosym}
\usepackage{wasysym}
\usepackage{amssymb}
\usepackage{capt-of}
\usepackage{hyperref}
\tolerance=1000
\usetheme{Antibes}
\usecolortheme{seagull}
\author{Sam Gwydir Chris Findeisen Martin Fracker Rafael Moreno Kyle Wilson}
\date{\textit{<2015-05-10 Sun>}}
\title{CSCE315 Final Project Presentation}
\hypersetup{
 pdfauthor={Sam Gwydir Chris Findeisen Martin Fracker Rafael Moreno Kyle Wilson},
 pdftitle={CSCE315 Final Project Presentation},
 pdfkeywords={},
 pdfsubject={},
 pdfcreator={Emacs 24.5.1 (Org mode 8.3beta)}, 
 pdflang={English}}
\begin{document}

\maketitle
\begin{frame}{Outline}
\tableofcontents
\end{frame}


\section{DickGrayson}
\label{sec:orgheadline1}
\begin{ABSTRACT}
In the final project students were tasked with creating a collection of tools
that allow a user to encrypt and decrypt messages using the RSA encryption
algorithm and embed and extract messages (either plaintext or ciphertext) from
BMP images and WAV audio files.
\end{ABSTRACT}

\begin{frame}[fragile,label=sec-1-1]{DickGrayson}
 \begin{block}{Introduction}
\emph{DickGrayson} is a collection of tools that allow a user to encrypt and decrypt
messages using the RSA encryption algorithm and embed and extract messages
(either plaintext or ciphertext) from BMP images and WAV audio files.
\end{block}

\begin{block}{Tools}
\begin{description}
\item[{\texttt{munchkincrypt}}] (aka \texttt{rsa-crypt}) RSA Encryption
\item[{\texttt{dorothy}}] (aka \texttt{rsa-attack}) RSA Attacks
\item[{\texttt{munchkinsteg}}] (aka \texttt{stego-crypt}) Steganography
\item[{\texttt{toto}}] (aka \texttt{stego-attack}) Steganography Attacks
\end{description}
\end{block}
\end{frame}

\section{Design}
\label{sec:orgheadline1}
\begin{frame}[fragile,label=sec-2-1]{Design Decisions}
 \begin{block}{DickGrayson}
\begin{description}
\item[{Target OS}] Linux \texttt{x86\_64} (\texttt{build.tamu.edu})
\item[{Compiler}] \href{https://gcc.gnu.org/}{GCC 4.9.2}
\item[{Language}] C++14
\item[{Build System}] \href{https://cmake.org}{CMake}
\item[{Numerics}] \href{https://gmplib.org}{GNU Multiple Precision Library}
\item[{Unit Testing}] \href{https://code.google.com/p/googletest/}{Google Test}
\item[{Continuous Integration}] \href{http://travis-ci.org}{travis-ci}
\item[{Code Coverage}] \href{http://coveralls.io}{coveralls}
\end{description}
\end{block}
\end{frame}
\begin{frame}[label=sec-2-2]{Implementation Decisions}
\begin{block}{RSA Encryption}
\begin{itemize}
\item Especially, explain what attacks you chose to make for both RSA and stego, and
how your two stego schemes work
\item Show proofs of Test-Driven Development (screenshots of failing / passing
tests, GitHub revision history, running test code in the demo, etc)
\end{itemize}
\end{block}
\begin{block}{RSA Attacks}
\begin{itemize}
\item Attacked using one general purpose attack, which would work on any rsa key, 
then created two more specific attacks that are efficient in some cases.
\end{itemize}
\end{block}
\end{frame}

\begin{frame}[label=sec-2-3]{Implementation Decisions (cont.)}
\begin{block}{Steganography}
\end{block}
\begin{block}{Steganography Attacks}
\end{block}
\end{frame}

\section{Demonstration}
\label{sec:orgheadline1}
\begin{frame}[label=sec-3-1]{Build System \& Tools}
\begin{itemize}
\item Travis
\item Coveralls
\end{itemize}
\end{frame}
\begin{frame}[fragile,label=sec-3-2]{\texttt{rsa-crypt}}
 \begin{itemize}
\item generate primes
\item generate public and private keys
\item encrypt and decrypt example messages
\end{itemize}
\end{frame}
\begin{frame}[fragile,label=sec-3-3]{\texttt{rsa-attack}}
 \begin{itemize}
\item factorization
\item common modulus
\item low exponent
\end{itemize}
\end{frame}
\begin{frame}[fragile,label=sec-3-4]{\texttt{stego-crypt}}
 \begin{block}{How it works}
Our tool munchkinsteg supports image and audio least significant bit (LSB)
steganography. The formats supported are Windows 8-bit BMP and PCM 16-bit WAV.
\end{block}
\begin{block}{Embedding message}
The data is broken up into bits and each bit is stored consecutively in the LSB
of the individual pixels and samples of the bmp and wav file respectively. A
null byte is stored using 8 additional pixels/samples to be used during
extraction as a termination symbol.
\end{block}
\end{frame}
\begin{frame}[fragile,label=sec-3-5]{\texttt{stego-crypt} (cont.)}
 \begin{block}{Extracting message}
The LSB of the individual pixels and samples of the bmp and wav file
respectively are extracted and concatenated into a string. This process
terminates once a null byte is reached.
\end{block}
\end{frame}

\begin{frame}[fragile,label=sec-3-6]{\texttt{stego-attack}}
 \begin{itemize}
\item Detection
\end{itemize}
\end{frame}
\section{Conclusion}
\label{sec:orgheadline1}
\begin{itemize}
\item Problems
\item Sucesses
\end{itemize}
\section{References}
\label{sec:orgheadline1}
\begin{frame}[label=sec-5-1]{Links}
\begin{description}
\item[{GCC 4.9.2}] \url{https://gcc.gnu.org}
\item[{GNU Multiple Precision Library}] \url{https://gmplib.org}
\item[{Google Test}] \url{https://code.google.com/p/googletest/}
\item[{travis-ci}] \url{https://travis-ci.org}
\item[{coveralls}] \url{https://coveralls.io}
\end{description}
\end{frame}
\end{document}