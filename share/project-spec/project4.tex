%% LyX 2.1.3 created this file.  For more info, see http://www.lyx.org/.
%% Do not edit unless you really know what you are doing.
\documentclass[english]{article}
\usepackage[T1]{fontenc}
\usepackage[latin9]{inputenc}
\usepackage{geometry}
\geometry{verbose,tmargin=1in,bmargin=1in,lmargin=1in,rmargin=1in}
\usepackage{babel}
\usepackage{graphicx}
\usepackage[unicode=true]
{hyperref}
 
\makeatletter
\@ifundefined{showcaptionsetup}{}{%
 \PassOptionsToPackage{caption=false}{subfig}}
\usepackage{subfig}
\makeatother
 
\begin{document}
 
\title{Project 4: Cryptography and Steganography}
 
 
\author{CSCE 315: Programming Studio}
 
 
\date{Spring 2015}
 
\maketitle
Your team has been hired by Munchkin Incorporated (Munchkin) to design
and develop a set of information assurance tools. To protect their
data and the data of their clients, Munchkin wants to be able to send
and receive data encrypted using the RSA cryptosystem. To further
protect their communications with valuable, high-profile clients,
Munchkin also wants to be able to hide their ciphertexts in images.
Munchkin's vibrant and popular social-media presence gives them plenty
of reason to be posting images on social media and other websites,
ostensibly so that their brand can maintain its recognizability and
relevance.
 
In addition to the tools for protecting the confidentiality of their
clients, Munchkin is requesting that you build tools for cracking
RSA and detecting images which are hiding secret information. The
reason for this, they tell you, is that they believe their direct
competitor, Elphaba International (Elphaba), will attempt to use ripoffs
of Munchkin's security tools to sabotage Munchkin's reputation in
the market. If Munchkin has tools for detecting and decrypting Elphaba's
knockoffs, they can expose Elphaba's scammy behavior and protect both
their clients and their brand.
 
Your contract with Munchkin Incorporated begins on April 6th. You
will use both Agile Programming and Test-Driven Development to design,
build, and deliver the 4 tools to Munchkin Incorporated on or before
Cinco De Mayo (the 5th of May).
 
 
\section*{Agile Programming and Test-Driven Development}
 
Your team will engage in 1-week Sprints. Before each Sprint, you will
take your Product Backlog and select from it the Sprint Backlog, a
list of functionality to work on for the upcoming Sprint. During each
Sprint, you will conduct ``daily'' Scrum meetings. Your team must
hold at least 4 Scrums per week. Each Scrum should last not more than
15-20 minutes. At the Scrum, three questions must be answered by each
member (the Sprint Status Check): \emph{What have you done since the
last scrum meeting? What has impeded your work? What do you plan on
doing between now and the next Scrum meeting?} The answers to these
questions should be recorded. Between daily Scrum meetings (ideally
at the end of each day), each team member should update the team's
Sprint Burndown Chart to show the amount of effort remaining on each
task in the Spring Backlog as well as the status of the task. At the
end of each 1-week Sprint, your team will hold a Sprint Review Meeting
(during a lab session or TA office hours) to demonstrate the new features
produced during the Sprint. At the Sprint Review Meeting, your team
will also submit the Backlogs, Burndown Charts, and Sprint Status
Checks to CSNet. After the Sprint Review Meeting, the process repeats,
with your team selecting a new list of functionality from the Product
Backlog (including functionality not completed during the previous
Sprint) to become the Sprint backlog for the next Sprint. As soon
as a tool is ready to be released to Munchkin for deployment (as determined
at the Sprint Review Meeting), you should submit the source code to
CSNet.
 
You \textbf{MUST }use Test-Driven Development (TDD). In your final
presentation to the board of Munchkin Incorporated (during the Final
Exam for CSCE 315), you must provide proof that your team used TDD
during development. This proof should be provided in the form of the
source code for the tests as well as documentation (e.g. timestamped
screenshots) showing that your team started with failing tests and
then developed code that would pass the tests. Munchkin Incorporated
believes so strongly in TDD that your contract with them includes
a penalty for not convincing them that you used TDD.
 
 
\section*{Important Dates}
\begin{enumerate}
\item April 6th: Begin Sprint \#1. Focus on defining requirements and assigning
team roles.
\item April 13th: Begin Sprint \#2. Submit progress report from Sprint \#1
to CSNet.
\item April 20th: Begin Sprint \#3. Submit progress report from Sprint \#2
to CSNet.
\item April 27th: Begin Sprint \#4. Submit progress report from Sprint \#3
to CSNet.
\item May 5th: Cinco De Mayo Delivery Day. Submit progress report from Sprint
\#4 and all deliverables to CSNet.
\item May 11th: Final Presentations at 10:30am in HRBB 124 (501-503), 3:30pm
in HECC 203 (504-506).
\end{enumerate}
 
\section*{Rubric}
\begin{itemize}
\item RSA cryptosystem: 15 pts
 
\begin{itemize}
\item Generate random primes with approximately $k$ bits, for $16\leq k\leq512$
\item Generate correct public and private keys for $k$-bit moduli, for
$32\leq k\leq1024$
\item Verify $D_{K}\left(E_{K}\left(M\right)\right)=E_{K}\left(D_{K}\left(M\right)\right)=M$
using team's own implementation
\item Verify using the \emph{openssl }command line tool:
 
\begin{itemize}
\item Correct encryption of a single block
\item Correct decryption of a single block
\item Correct encryption of multiple blocks
\item Correct decryption of multiple blocks
\end{itemize}
\end{itemize}
\item Attacks on RSA: 15 pts
 
\begin{itemize}
\item At least 3 working attacks on RSA
\end{itemize}
\item LSB image stegosystem: 15 pts
 
\begin{itemize}
\item Correct embedding of bits in the 1-LSB plane
\item Correct extraction of bits from the 1-LSB plane
\item Correct embedding of bits in the 2-LSB plane
\item Correct extraction of bits from the 2-LSB plane
\item Colors: Grayscale and RGB
\item Correct PSNR reported
\end{itemize}
\item Attacks on LSB image stego: 15 pts
 
\begin{itemize}
\item At least 3 working attacks on LSB image stego
\end{itemize}
\item Weekly Sprint progress reports: 20 pts
 
\begin{itemize}
\item 5 points per week
\item Include Backlogs, Burndown Charts, and Sprint Status Checks
\end{itemize}
\item Final Presentation: 20 pts
 
\begin{itemize}
\item Demonstrate tools for
 
\begin{itemize}
\item Encryption, Decryption, Embedding, Extracting, Cryptanalysis, Steganalysis
\end{itemize}
\item Proof of Test-Driven Development
\item Do not exceed 10 minutes
\end{itemize}
\end{itemize}
 
\section*{RSA Encryption}
 
RSA is a public-key cryptosystem proposed in 1977 by Rivest, Shamir,
Adleman. It is the most successful public-key cryptosystem and is
based on the idea that the factorization of integers is a hard problem.
 
A summary of the RSA algorithm:
\begin{enumerate}
\item Generate distinct large primes $p$ and $q$ and compute $n=pq$ and
$\phi(n)=(p-1)(q-1)$.
\item Pick public key exponent $e$ such that $1<e<\phi(n)$ and $e$ and
$\phi(n)$ are coprime.
\item Compute private key exponent $d$ such that $ed\equiv1\pmod{\phi(n)}$.
\item Release $\left\{ n,e\right\} $ as the public key. Keep $\left\{ p,q,d\right\} $
as secret key.
\item To encrypt a message, first convert the plaintext message $M$ to
a number $m$ such that $0\leq m<n$. Then, compute the ciphertext
as $c\equiv m^{e}\pmod{n}$.
\item To decrypt a ciphertext, compute $m\equiv c^{d}\pmod{n}$ and then
convert the number $m$ into the plaintext $M$ by reversing the procedure
which converted $M$ to $m$.
\end{enumerate}
You must create a command line tool named \emph{munchkincrypt} which
implements the RSA cryptosystem and related functionality. At minimum,
your tool must implement the following functionality:
\begin{itemize}
\item generate a prime number of a given approximate bit-length
\item generate a public and private key pair of a given approximate bit-length
\item encrypt cleartext using a given public key
\item decrypt ciphertext using a given private key
\end{itemize}
You are encouraged to implement functionality in addition to the above
list of minimum required functionality.
 
 
\section*{Cryptanalysis of RSA}
 
The security of RSA depends on several factors. Fundamentally, RSA
relies on the problem of factorizing integers being a hard problem.
If Eve can factor Alice's public modulus $n$, then she can easily
compute Alice's private exponent $d$ and can therefore read all of
Alice's mail and forge Alice's digital signature. If integer factorization
turns out to be easy (i.e. if there exists a polynomial time algorithm
for factoring), the RSA is completely useless. Another weakness of
RSA (and of every cryptosystem) is that the implementation and application
of the cryptosystem can be attacked directly. If not implemented and
used correctly, RSA can be a piece of cake to break. Here are just
a few ways that RSA can be broken due to implementation mistakes.
\begin{itemize}
\item If the modulus $n$ is small (less than 512 bits), then a desktop
PC and a good factoring algorithm can factor it in a matter of days.
\item If the primes $p$ and $q$ that make up the modulus were created
in a way that makes them likely to be close together, and therefore
close to $\sqrt{{n}}$, then $n$ can be factored using Fermat factorization.
\item If either $p-1$ or $q-1$ has only small prime factors, then $n$
can be factored using Pollard's $p-1$ algorithm
\item If $q<p<2q$ and $d<\frac{n^{1/4}}{3}$, then $n$ can be factored
efficiently using Wiener's theorem.
\item If the public encryption exponent $e$ is small (e.g. $e=3$) and
the message $m$ is small, so that $m^{e}<n$, then $m$ can be recovered
by finding the $e$-th root of the ciphertext.
\item If the same message is encrypted using the same public exponent, but
different public moduli, then the message can be recovered using the
Chinese remainder theorem.
\item If the random number generator used to generate the prime factors
is not sufficiently random, a large-enough collection of public keys
generated by it will contain pairs of moduli which can be factored
by Euclid's algorithm.
\end{itemize}
You must create a command line tool named \emph{dorothy }which implements
at least three attacks on the RSA cryptosystem. You may choose any
three of the above attacks, or you may find/design others.
 
 
\section*{LSB Image Stego}
 
Steganography (or Stego, for short) is the art and science of hiding
data so that only the sender and intended receiver are aware of the
existence of the data. It is complementary to cryptography. While
cryptography protects \emph{what} Alice sends to Bob by obfuscating
the contents of her messages, steganography protects \emph{when} and
\emph{if} Alice sends data to Bob by disguising her messages as innocuous
channel usages (e.g. as images, audio, text, etc.). As in cryptography,
the security of a stegosystem should not rely on the secrecy of the
method. You must assume that the adversary knows the system. The strength
of the stegosystem is determined by how difficult it is for a passive
observer to distinguish between legitimate cover-objects which are
not hiding data and stego-objects which contain or encode hidden data.
The more difficult it is to detect the hidden data, the more secure
the stegosystem.
 
One of the most basic kinds of steganography deals with images. An
image is made up of many tiny picture elements, called pixels. Each
pixel is a single color. The color is encoded as a number. For grayscale
images, each pixel is 8-bits long and represents one of 256 possible
shades of gray. For RGB images, each pixel is 24-bits long and represents
256 shades each of red, green, and blue (you can think of RGB pixels
as three 8-bit pixels, one for red, one for green, and one for blue).
The low order bits of each pixel control differences in shades of
color which are beneath the perceptual threshold of the human visual
system. Therefore, these bits can be modified without any perceptible
image degradation. This property can be exploited for data hiding
by replacing the least significant bits of each pixel with the bits
of the secret data.
 
\begin{figure}[h]
\subfloat{\protect\includegraphics[scale=0.33]{lena}}\hfill{}\subfloat{\protect\includegraphics[scale=0.33]{lena2}}\protect\caption{One of these is the original image of Lena and the other is hiding
32KB of secret data.}
\end{figure}
 
 
You must create a command line tool named \emph{munchkinsteg }which
implements LSB image stego embedding and extraction for grayscale
and RGB images in BMP format. Your tool must support at least 1-LSB
(use only the least significant bit) and 2-LSB (use the two least
significant bits) embedding, but may support other modes in addition
to these. Your tool must report the PSNR (peak signal to noise ratio)
of the stego-image.
 
 
\section*{Steganalysis and Countermeasures}
 
Since the LSB-plane is beneath the human perceptual threshold, a human
adversary will not be able to detect the modification of an image
by sight alone, even if she has both the original and modified versions
(and does not know which is which). However, the deficiency of the
human visual system can be compensated for by some cleverness and
computational power. Any modification to an original object will introduce
some amount of distortion. Given an accurate model of the cover media
(e.g. how neighboring pixels in images relate to each other), this
distortion can be detected and used to distinguish between authentic
cover-objects and modified stego-objects. There are also some more
straightforward, but less clever, ways to determine if an image has
been modified. In cases where detection may be too difficult, or not
of primary importance, there are also countermeasures which may be
deployed to disrupt or eliminate the ability for an image to hide
data. Some ideas for detection techniques and countermeasures are
given below.
\begin{itemize}
\item If the original image, or a signature of the original image (e.g.
MD5 hash), is known, every version can be compared to that original
to determine if the image was modified.
\item If the hidden data is known to be cleartext (unencrypted data), then
the data can be extracted from a suspect image and tested for structure.
If an image is clean, the extracted data will be random, but if the
extracted data has an identifiable structure (e.g. ASCII text), then
it can be concluded that the image is hiding data.
\item A particularly clever technique for detecting LSB stego in images
is Regular-Singular Analysis\\
(\href{http://www.ws.binghamton.edu/fridrich/Research/acm_2001_03.pdf}{http://www.ws.binghamton.edu/fridrich/Research/acm\_{}2001\_{}03.pdf}).
RS analysis is able to detect data hidden at rates as low as 0.05
bits per pixel by looking at the differences between neighboring pixels
to estimate the amount (or \emph{length}) of data hidden in the image.
If the \emph{length} is greater than some threshold (e.g. 3\%), the
image is classified as steganographic.
\item RS Analysis is a special case of Sample Pairs Analysis\\
(\href{http://www.ece.mcmaster.ca/~sorina/papers/LSBfinalTSP.pdf}{http://www.ece.mcmaster.ca/$\sim$sorina/papers/LSBfinalTSP.pdf}).
Both methods compare a kind of noise in the sample image to an expected
level and type of noise in the cover object. Too much noise, or the
wrong kind of noise, indicates that the image has likely been modified.
\item If you can't beat 'em, make a giant mess. Since LSB-stego hides data
in a place where changes do not affect perceived image quality, the
entire LSB-plane can be randomized to destroy any hidden data also
without affecting perceived image quality.
\end{itemize}
You must create a command line tool named \emph{toto} which implements
at least three attacks on LSB image steganography systems. You may
choose any three of the above attacks, or you may find/design others.
 
\newpage{}
 
 
\subsubsection*{Reminder about Resource-Intensive Jobs}
 
If you run a resource-intensive program on any CS server other than
\texttt{compute.cse.tamu.edu}, your CS account may be terminated for
hogging the machine!
 
 
\subsubsection*{Libraries and Other Resources}
 
You \emph{may} use libraries for handling large numbers and performing
basic mathematical operations on them. You \emph{may} use libraries
for manipulating image data. You \emph{may} consider using/contributing
to to Virtual Steganographic Laboratory (\href{http://vsl.sourceforge.net/}{http://vsl.sourceforge.net/})
or Digital Invisible Ink Toolkit\\
(\href{http://diit.sourceforge.net/}{http://diit.sourceforge.net/})
projects. You \emph{may} use libraries for unit testing. For now,
that is it. If there is a library or resource you want to use and
it is not listed here, you must ask Dr. Ritchey about it before you
may use it. Do not ask about using libraries or code that solve the
main problems of this project, the answer will be ``No''.
 
 
\subsubsection*{GitHub}
 
You must use GitHub. Your repository must be set up and active before
April 12th. The TAs and the Instructors must have read access to your
repository by April 12th. Your repository must be private until May
5th. Your repository must be public after May 11th.
 
 
\subsubsection*{Final Report and Presentation}
 
The board of Munchkin Incorporated expects to receive a final report
of your work and a brief presentation demonstrating the tools you
built. The report should include details about how you solved each
problem in this project, especially with regard to Test-Driven Development
and Agile Programming. Your report also should also include the division
of labor, specifying who did what and the value of that contribution
to the overall project. Attached to the report, you should submit
copies of your weekly Sprint progress reports, complete with Backlogs,
Burndown charts, and Sprint Status Checks. Your presentation during
the final exam should last no more than 10 minutes and should clearly
demonstrate your usage of Test-Driven Development and Agile Programming,
as well as the correct operation of your tools. Stay true to the Agile
methodology, do not submit or demo something which is not yet finished.
 
 
\subsubsection*{Individual Scores}
 
Your individual score will be computed as $\min\left(teamScore\times\sqrt{\frac{yourContribution}{100/\left|group\right|}},110\right)$.
The team score will be calculated by the rubric given above. Your
contribution will be computed as the average of your contribution
percentage as given in the report and your contrubution percentages
as reported by your team members in the post-project survey.
 
 
\subsubsection*{Need More Information?}
 
This project specification is by no means intended to instruct you
on how to implement the tools required by this project. You will need
to attend lecture, ask questions, and do some independent study is
order to complete this project. If you feel like there is some piece
of information you are missing and cannot find it in this document,
have not learned it in lecture, do not understand the material you
have found online or in a textbook, or do not even know where to look
to find it, go to Piazza and ask your question there.
 
 
\subsubsection*{Important Legal Notice}
 
The techniques you are learning in this course and project are for
educational purposes only and are not to be used for any unethical
purposes. Please be responsible and use your powers for good.
\end{document}
